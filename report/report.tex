%%%%%%%% DATA LITERACY 2025 LATEX PROJECT TEMPLATE FILE %%%%%%%%%%%%%%%%%
%%% Based on the 2025 ICML template, available at https://icml.cc/Conferences/2025/AuthorInstructions %%%

\documentclass{article}

% Recommended, but optional, packages for figures and better typesetting:
\usepackage{microtype}
\usepackage{graphicx}
\usepackage{subfig}
\usepackage{booktabs} % for professional tables

\usepackage{tikz}
% Corporate Design of the University of Tübingen
% Primary Colors
\definecolor{TUred}{RGB}{165,30,55}
\definecolor{TUgold}{RGB}{180,160,105}
\definecolor{TUdark}{RGB}{50,65,75}
\definecolor{TUgray}{RGB}{175,179,183}

% Secondary Colors
\definecolor{TUdarkblue}{RGB}{65,90,140}
\definecolor{TUblue}{RGB}{0,105,170}
\definecolor{TUlightblue}{RGB}{80,170,200}
\definecolor{TUlightgreen}{RGB}{130,185,160}
\definecolor{TUgreen}{RGB}{125,165,75}
\definecolor{TUdarkgreen}{RGB}{50,110,30}
\definecolor{TUocre}{RGB}{200,80,60}
\definecolor{TUviolet}{RGB}{175,110,150}
\definecolor{TUmauve}{RGB}{180,160,150}
\definecolor{TUbeige}{RGB}{215,180,105}
\definecolor{TUorange}{RGB}{210,150,0}
\definecolor{TUbrown}{RGB}{145,105,70}

% hyperref makes hyperlinks in the resulting PDF.
% If your build breaks (sometimes temporarily if a hyperlink spans a page)
% please comment out the following usepackage line and replace
% \usepackage{icml2023} with \usepackage[nohyperref]{icml2023} above.
\usepackage{hyperref}


% Attempt to make hyperref and algorithmic work together better:
\newcommand{\theHalgorithm}{\arabic{algorithm}}

\usepackage[accepted]{icml2025}

% For theorems and such
\usepackage{amsmath}
\usepackage{amssymb}
\usepackage{mathtools}
\usepackage{amsthm}

% if you use cleveref..
\usepackage[capitalize,noabbrev]{cleveref}

% Todonotes is useful during development; simply uncomment the next line
%    and comment out the line below the next line to turn off comments
%\usepackage[disable,textsize=tiny]{todonotes}
\usepackage[textsize=tiny]{todonotes}


% The \icmltitle you define below is probably too long as a header.
% Therefore, a short form for the running title is supplied here:
\icmltitlerunning{Project Report Template for Data Literacy 2025}

\begin{document}

\twocolumn[
\icmltitle{Deconstructing a Decade of Migration Discourse in the European Parliament}

% It is OKAY to include author information, even for blind
% submissions: the style file will automatically remove it for you
% unless you've provided the [accepted] option to the icml2023
% package.

% List of affiliations: The first argument should be a (short)
% identifier you will use later to specify author affiliations
% Academic affiliations should list Department, University, City, Region, Country
% Industry affiliations should list Company, City, Region, Country

% You can specify symbols, otherwise they are numbered in order.
% Ideally, you should not use this facility. Affiliations will be numbered
% in order of appearance and this is the preferred way.
\icmlsetsymbol{equal}{*}

\begin{icmlauthorlist}
\icmlauthor{Giorgi Gogelashvili}{equal}
\icmlauthor{Samia Haque}{equal}
\icmlauthor{Jakob Kleine}{equal}\\
\icmlauthor{Dennis Stroh}{equal}
\icmlauthor{Quirin Unterguggenberger}{equal}
\end{icmlauthorlist}

% fill in your matrikelnummer, email address, degree, for each group member
% \icmlaffiliation{first}{Matrikelnummer 12345678, MSc Machine Learning}
% \icmlaffiliation{second}{Matrikelnummer 12345678, MSc Computer Science}
% \icmlaffiliation{third}{Matrikelnummer 12345678, MSc Media Informatics}
% \icmlaffiliation{fourth}{Matrikelnummer 12345678, MSc Medical Informatics}
% \icmlaffiliation{fifth}{Matrikelnummer 12345678, MSc QDS}

% put your email addresses here. You can use initials to save space, 
% e.g. if you are called Max Mustermann, you can use \icmlcorrespondingauthor{MM}{max.mustermann@uni-tuebingen.de}
% DO USE YOUR UNIVERSITY EMAIL ADDRESS!

% for the Data Literacy report, to save space, you can here list the student email address of one author, who is willing to be contacted about this work in the future (e.g. in case we would like to use your report as an example for future course iterations)
\icmlcorrespondingauthor{GG}{giorgi.gogelashvili@student.uni-tuebingen.de} 
% \icmlcorrespondingauthor{Initials2}{first2.last2@uni-tuebingen.de}
% \icmlcorrespondingauthor{Initials3}{first3.last3@uni-tuebingen.de}
% \icmlcorrespondingauthor{Initials4}{first4.last4@uni-tuebingen.de}
% \icmlcorrespondingauthor{Initials5}{first5.last5@uni-tuebingen.de}

% You may provide any keywords that you
% find helpful for describing your paper; these are used to populate
% the "keywords" metadata in the PDF but will not be shown in the document
\icmlkeywords{Machine Learning, ICML}

\vskip 0.3in
]

% this must go after the closing bracket ] following \twocolumn[ ...

% This command actually creates the footnote in the first column
% listing the affiliations and the copyright notice.
% The command takes one argument, which is text to display at the start of the footnote.
% The \icmlEqualContribution command is standard text for equal contribution.
% Remove it (just {}) if you do not need this facility.

%\printAffiliationsAndNotice{}  % leave blank if no need to mention equal contribution
\printAffiliationsAndNotice{\icmlEqualContribution} % otherwise use the standard text.

Motivated by the rise of populism in Europe since the late 1990s, this study investigates ideological shifts in European Parliament (EP) speeches using natural language processing. Drawing on the novel ParLawSpeech dataset \citep{SDN-10.7802-2824} which contains 574,199 speeches from 1999 to 2024 alongside metadata on speaker identity, we use sentence embedding models to examine the semantic content and emotional tone of parliamentary debates over time.

We expect that speech embeddings will form clusters reflecting party affiliation and ideological alignment. In step with recent political developments, we further hypothesize an increase in negative sentiment within the immigration debate among centrist and right-wing groups, accompanied by growing semantic similarity between these two factions over the past two decades. Finally, we test whether established migration-related narratives associated with right-wing populism can be identified in parliamentary discourse and how their prevalence has developed over time.



The continued success of right-wing populist parties in the 21st century is widely regarded as a major threat to European democracy and integration \citep{fossum_what_2023,rummens_populism_2017}. Populist rhetoric is commonly defined as constructing an antagonism between a `pure people' and a `corrupt elite' \citep{mudde_populist_2007}. Right-wing populism is also closely tied to the issue of immigration. Parties of this ideology have played a central role in the increasing politicisation of immigration \citep{hutter_politicising_2022}, which represents a crucial factor for their political success \citep{kende_xenophobia_2020}. Over the past decade, immigration has become an increasingly salient issue in European election campaigns \citep{dekeyser_elections_2023} as well as in media coverage \citep{greussing_shifting_2017}. [introduce agenda setting here?]

Electoral gains of populist parties have manifested in significant changes of parliamentary discourse \citep{schwalbach2023talking}. A recent quantitative analysis of EP speech embeddings has identified a gradual increase in emotional rhetoric from 1999 to 2022, with right-wing populist groups leading the trend \citep{subtil_verger_2024}. In the German national parliament, an LLM-based study has revealed increasing anti-solidarity messaging around immigration, not only for right-wing, but also christian-conservative and liberal parties \citep{kostikova_llm_2025}. This trend begins around 2015, which marks the onset of the so-called `refugee crisis' \citep{brueckner_crisis_2026}.

More fine-grained analyses of the migration discourse have revealed the use of common underlying narratives, defined as `selective depictions of reality' and `patterns of interpretation' through which the issue is relayed to the public. Social media posts from populist leaders commonly employ anti-immigrant frames like `immigrants take our jobs' or anti-establishment narratives such as `our sovereignty is under threat' \citep{seiger_navigating_2025}.

These previous successes in deconstructing and quantifying highly impactful political trends inspired the present investigation of how the growing prominence of right-wing populism and immigration as a salient political issue manifest in EP debates, with potential implications for broader societal discourse and legislative outcomes. All parliamentary speeches between 2014 and 2024 as recorded by the ParlLawSpeech dataset inform the analyses.

We employ unsupervised topic modeling to break down the prevalence of migration debates relative to other topics over the last decade, and provide evidence of agenda-setting strategies employed by right-wing populist groups (\Cref{sec:methods}). By embedding migration-labeled speeches into vector representations, we examine the semantic dimensions along which party groups can be differentiated and identify [wording too strong?] an increased use of previously identified anti-immigration narratives by right-wing groups [compared to moderate factions?] (\Cref{sec:results}).

% previous versions:

% Building on these foundations, we apply computational methods to European Parliament debates to analyze migration discourse. We combine topic modeling to measure issue salience, semantic embeddings to map ideological positioning, and narrative detection to quantify the adoption of populist rhetorical frames across party groups.

% Speeches are first classified into topics using Latent Dirichlet Allocation (LDA) to identify immigration-related debate and to estimate its prominence over time. Analyses of the distribution of migration-related speeches across predefined debate agendas provide quantitative evidence consistent with agenda-setting strategies employed by right-wing populist groups. With the use of speech embeddings, we examine the semantic dimensions along which party groups can be differentiated and find evidence for an increased use of previously identified anti-immigration narratives by right-wing groups compared to moderate factions.

% instructions: 

% Motivate the problem, situation or topic you decided to work on. Describe why it matters (is it of societal, economic, scientific value?). Outline the rest of the paper (use references, e.g.~to \Cref{sec:methods}: What kind of data you are working with, how you analyse it, and what kind of conclusion you reached. The point of the introduction is to make the reader want to read the rest of the paper.


\section{Data and Methods}\label{sec:methods}

    % In this section, describe \emph{what you did}. Roughly speaking, explain what data you worked with, how or from where it was collected, it's structure and size. Explain your analysis, and any specific choices you made in it. Depending on the nature of your project, you may focus more or less on certain aspects. If you collected data yourself, explain the collection process in detail. If you downloaded data from the net, show an exploratory analysis that builds intuition for the data, and shows that you know the data well. If you are doing a custom analysis, explain how it works and why it is the right choice. If you are using a standard tool, it may still help to briefly outline it. Cite relevant works. You can use the \verb|\citep| and \verb|\citet| commands for this purpose \citep{mackay2003information}.

    % This is the template for a figure from the original ICML submission pack. In lecture 10 we will discuss plotting in detail.
    % Refer to this lecture on how to include figures in this text.
    % 
    % \begin{figure}[ht]
    % \vskip 0.2in
    % \begin{center}
    % \centerline{\includegraphics[width=\columnwidth]{icml_numpapers}}
    % \caption{Historical locations and number of accepted papers for International
    % Machine Learning Conferences (ICML 1993 -- ICML 2008) and International
    % Workshops on Machine Learning (ML 1988 -- ML 1992). At the time this figure was
    % produced, the number of accepted papers for ICML 2008 was unknown and instead
    % estimated.}
    % \label{icml-historical}
    % \end{center}
    % \vskip -0.2in
    % \end{figure}

    \subsection{Dataset description}
    To analyze how migration is discussed in the EP, we rely on the \textit{ParlLawSpeech} (PLS) dataset \citep{SDN-10.7802-2824}, 
providing us with a large-scale corpus of more than half a million verbatim parliamentary speeches alongside rich metadata, covering European legislatures between 1999 and 2024.

We restrict our analyses to the last two complete legislative periods (2014–2024), as the 2015/2016 “refugee crisis” marks a qualitative shift in the nature and salience of 
migration-related debates that could bias topic-model estimation. This shift is reflected in the proportion of migration-labeled speeches, which remains around 1\% prior to 2014 but rises to over 2.5\% thereafter.

We further enrich the PLS corpus with additional speaker-level metadata obtained from the EP’s Open Data Portal API, in particular the national party affiliation of 
each Member of the European Parliament (MEP) at the time a speech was delivered. 

To contextualize and validate patterns observed in parliamentary speech, we also draw on expert's perceptions of party positions from the \textit{Chapel Hill Expert Survey} 
(CHES) of \citet{rovny_CHES_2024}. CHES is a long-running study, surveying hundreds of country experts who estimate the ideological positions of national political parties across Europe on a 
wide range of dimensions, such as general left–right ideology, European integration, and issue-specific policy stances, e.g. experts are asked to place parties on a scale from 0 ("strongly favors liberal
policy on immigration") to 10 ("strongly favors restrictive policy on immigration).
Since 1999, the survey has been conducted roughly every four years, including 2014, 2019 and 2024, offering repeated cross-sectional measurements of 
party positions that enable the study of ideological change over time.
While parliamentary speeches capture how parties communicate and frame political issues, CHES provides an independent, expert-based assessment of where parties are 
positioned in ideological space. Combining the two allows us to triangulate latent patterns in political discourse against an external benchmark that is not derived 
from the same textual data. Importantly, CHES scores are not used to label or classify individual speeches, but instead serve as contextual variables against which aggregated 
speech patterns can be compared.

% TODO specific numbers/facts on CHES<->PLS merging? (how many rows enriched etc.)

%More recent CHES waves include detailed assessments of parties’ positions on immigration and multiculturalism, 
%making the dataset particularly valuable for studies of migration-related political conflict.
% 
%
% By employing Latent Dirichlet Allocation (TODO verweis sec), we identify speeches related to the topic of migration, and after only including the legislative periods between 2014 and 2024
% this leaves us with 9,705 datapoints in total.

% After only including speeches of legislative terms between 2014 and 2024, this leaves us with XXXX datapoints in total.

% -> put this at the end of the preprocessing section?


% This belongs to preprocessing:
% Most score dimensions in the CHES dataset range from 0-10, except for a few survey items (e.g. having a scale of 1-7) which we recalibrated to match the 0-10 range, as done in
% [here](https://onlinelibrary.wiley.com/doi/epdf/10.1111/ajps.12115?saml_referrer)

%%% OLD TEXT
%We are using the novel \textit{ParlLawSpeech} (PLS) dataset from Schwalbach et al. 2025 for the investigation of our study. 
%It contains more than 570,000 plenary speeches from legislative periods of the European parliament (EP) between 1999 and 2024.
%The authors also provide (partially) machine translated text in English for about 40\% of the speeches, since the EP stopped providing official translations around the end of 2012. 
%Furthermore, the dataset contains metadata on the speakers and the speeches given,
%e.g. date and agenda item under which the speech was given, if submission was in written form and/or from multiple \textit{members of parliament} (MEPs), or the speaker's 
%party affiliation (referring to European political parties/groups), among other. We further enriched
%the dataset with metadata accessible from the public API of the EP's "Open Data Portal", in particular the national party affiliations of each speaker (by using the \textit{EP-ID} of the 
%respective MEPs). This allowed us to link the PLS dataset with the \textit{Chapel Hill Expert Survey} (CHES)
%from Rovny, Bakker et al. 2025. The CHES dataset estimates party positioning on European integration, ideology (e.g. left/right) and policy issues for national parties in all member states of 
%the European Union (EU). The study surveyed hundreds of experts roughly every four years
%between 1999 and 2024 and more recently (**TODO**: since when???) also includes ratings of non-EU policy issues such as immigration or anti-elite rhetoric (**TODO:** which are relevant in 
%particular?) Assuming that the ideological orientation of a speaker's affiliated national party
%roughly reflects his own position, the CHES data set could help us to better control our analyses, as membership of a European party (group) presumably allows for less detailed/granular 
%statements/assumptions.
%We restrict our analyses to the last two complete legislative periods (2014–2024), as the 2015/2016 “refugee crisis” marks a qualitative shift in the nature and salience of 
%migration-related debates that could bias topic-model estimation. This shift is reflected in the proportion of migration-labeled speeches, which remains around 1\% prior to 2014 but rises to over 2.5\% thereafter.


% 2015 and 2016 anomaly 
\textbf{2015 and 2016 anomaly} In 2015 and 2016, there is a drastic increase in the number of speeches compared to the other years (72,964 in 2015 and 16 on average, compared to 10,098 in the other years on average). 
This difference can be explained with a rule-change that was adopted by the parliament at the end of 2016, discontinuing so-called 'written declarations' that allowed party members to hand in short expressions of 
opinion on a certain issue [TODO ref]. Omitting all 'written' speeches (TODO: explain what written flag means) can normalize the number of speeches per year. However, for the general analysis, we keep written declarations as 
part of the dataset because they can contain relevant stances of the parties on political issues. Since written declarations could be co-signed by multiple speakers, they can appear duplicated in the dataset. 
These duplicated items were removed, keeping only the first instance of a duplicated speech.


    \subsection{Data preprocessing}

    % \input{sections/2_data_methods/preprocessing/duplicates.tex}
    \textbf{Translation}
To keep the speeches most comparable in the embedding space, we use English translations instead of the orignal speeches. 
Until the year X (TODO), the Parllaw dataset includes a machine translation for each speech. The remaining X (TODO) translations were created using Gemini 2.5-flash \citep{gemini_2025}. We tested its translations on a random sample of speeches that had already been translated by Parllaw and checked that 
% We checked that Gemini \emph{1)} did not re-formulate speeches that were already in English and that \emph{2)} its translations are comparable to Parllaw's  
% in the embedding space. For this, we tested its translations 
Gemini \emph{1) } preserved speeches which were already in English
% \footnote 
    % {(TODO add quantifier) Since we created translations before extensively cleaning the dataset, some English speeches included bracketed language flags that led to Gemini re-translating the English speeches.
    % These reformulations are however almost identical to the original speech. Therefore, we accepted those instances where Gemini failed to recognize English texts.} 
and \emph{2) } created translations whose embeddings are very similar to Parllaw's for non-English speeches (bootstrapped 0.95 confidence interval of mean cosine similarity: 0.969, n=1001). 
Thus, we assume that Gemini's and Parllaw's translations are similar enough to fill in the missing translations with Gemini's, and conduct our analysis under the assumption that all translations stem from the same source. 
% However, we note that the mixture of two translation approaches might nevertheless introduce a bias to our dataset, that we have to check for. (TODO: did we check for that)

    
\textbf{Removing Commentary} (TODO)
We detect high amount of superfluous commentary in transliterated speeches:  markers of the original language, background incidents, and procedural notes. These markers might be source of unwanted bias, which we want to avoid. Fortunately they are predominantly located within parentheses and can be easily removed with rule-based methods. We also observe substantial redundency in the openning and closing sections of the speeches.
These sections follow similar rhetorical structures but exhibit substantial lexical variation. To identify low-impact sentences we use TF-IDF algorithm to score the amount of information they contain.
We construct separate corpora for opening and closing sentences, and an average TF-IDF score is computed for each sentence. Two independent raters annotated a sample of 100 low-scoring sentences for informational relevance. We then fit a logistic regression linking TF–IDF score percentiles to these annotations and used the model to estimate the threshold required to achieve 95\% classification accuracy.


    \begin{figure}[ht]
\vskip 0.2in
\begin{center}
\centerline{\includegraphics[width=\columnwidth]{"fig/fig1_lda.pdf"}}
\caption{LDA results. \textbf{Top:} Number of speeches for selected topics in EP debates, divided by total number of speeches per year. Topic labels were created manually based on most-frequent topic words. See repository for an interactive version with all topics.
\textbf{Bottom:} Absolute number of migration speeches by parliamentary group. Written speeches were discontinued in 2017.}\label{fig:fig1_lda}
\end{center}
\vskip -0.2in
\end{figure}

    \subsection{Methods}


    
\subsubsection{Topic Modelling with LDA}

To isolate migration-related discourse from all speeches, we use \emph{Latent Dirichlet Allocation} \citep{blei2003latent}. LDA models each speech as a probabilistic mixture of a predefined number of topics, where each topic is defined by a distribution over words. For every speech, the model estimates the probability of belonging to each topic.

We evaluated multiple model specifications and selected the model based on topic coherence (final score: 0.56) and manual inspection of topic interpretability. The selected model comprises 30 topics, one of which assigned highest probabilities to the words \emph{refugee}, \emph{border}, and \emph{migration}. Speeches were categorized as migration-related if they had an above-threshold probability for this topic ($n = 9705$).

The threshold was determined by two raters sampling 100 speeches from the probability range where the cutoff was expected based on initial tests %([0.20, 0.35])
and classifying whether they were migration-related. Using receiver operating characteristic analysis, we identified the threshold that minimized the difference between true and false positive rates. %(prob = 0.25).

% Why LDA and not something else?
% (TODO why LDA?)
% Description LDA
% LDA is a probabilistic topic model. 
% It assumes that in the analyzed corpus (here: the collection of speeches) there is a set of topics, which are probability distributions over all words in the corpus. 
% It considers each document (here: a speech) as a bag of words that were sampled from these topics. 
% For example, if a topic had high probabilities for the words "fish", "net", "water", then documents covering "fishing" would (under LDA's assumptions) have a high probability of being labelled as that topic. 
% Our LDA


% We tested different parameters (numer of topics, number of iterations over the dataset) and compared 
% the resulting topic coherence (TODO explanation) and the fidelity of the topics through manual inspection.
% Our final model contains 30 topics (for 10 iterations) --- one of which assigns highest probabilities to the words X, Y, Z (TODO words), which we call "migration topic".

% threshold


    \subsubsection{Quantifying agenda setting}
% - control of agenda => political power (ref. to oxford paper?)
% - finding in germany: political right has influence over what other parties talk about. 
% seeing that parties infleunce the discourse, we ask ... 
Seing that parties influence and shape the debate \citep{saldivia_gonzatti_agenda_2026}, we investigate whether parties systematically emphasize migration in different contexts, and thereby ``put migration on the agenda''. 
% A phenomenon studied in the literature is \emph{agenda setting}, which describes a political acteur's strive to place their topic of interest in the discourse [TODO: ref]. In the context of parliamentary debate in the EU, it is interesting to ask: do parties try to put migration on the agenda? 
For this purpose, we look at debates where all migration-related speeches stem from the same party block, indicating that migration is fed into the discussion by that group despite not being the main topic. We define such a debate as an instance of agenda-setting. We assign each debate a `true' topic, which is the topic with highest average probability among its speeches.
    \subsubsection{Semantic Embeddings}
Semantic embeddings have been widely used in political text analysis \citep{Miok2024, Nanni2021, Rudkowsky2018}. Our aim is to capture patterns in how different political groups address migration. We select candidate embedding models from the MTEB leaderboard \citep{enevoldsen2025mmtebmassivemultilingualtext}, based on overall performance and parameter count. Final model selection is based on (i) intra- and interparty cosine similarities, (ii) predictive performance of a logistic regression model with political affiliation as our target variable, and (iii) Kmeans clustering quality measured by homogenity and completeness. 
Based on these metrics we have selected google/embeddinggemma-300m \citep{embedding_gemma_2025} as our final embedding model, all analysis using semantic embeddings are conducted with this model. 

A key concern is that general-purpose semantic embeddings may be primarly capturing stylistic and topical variations and subsequently political group ideologies influence on the embeddings might be neglegible. We test whether intra- and interparty similarity distributions differ substantially with a two-sample Kolmogorov-Smirnov test for each candidate model.
We apply Bonferroni correction accross 8 models ($\alpha = 0.05$, $\alpha\mbox{*} = 0.05 / m$). All models showed significant distributional differences, with test statistic $\mathcal{D}$ ranging between 0.058 and 0.1.

Dimensionality reduction has been used to ascertain parties ideological shift over time and to reveal underlying political dimension with word associations for each reduced axis  \citep{Rheault2020-mr}. Exploratory analysis showed that, although party influence is present, it is not the defining factor of our semantic embeddings. To better understand how party affiliations manifest in the vector space, we aim to identify a subspace of the embedding space in which political and ideological differences become more salient.
To this end, we employ Partial Least Squares (PLS). PLS allows us to find directions in the embedding space that are maximally associated with party labels, making it suitable for uncovering latent political dimensions that are not necessarily dominant in the overall variance of the data.

The prevalence of established migration-related rhetoric was assessed using semantic search in a shared embedding space. We used 30 migration narratives identified in a recent report by the European Commission’s Joint Research Centre \citep[p.130]{seiger_navigating_2025}, organized into four broader ``supernarratives.'' Each narrative was represented by a short descriptive sentence and embedded using the model’s built-in ``retrieval-query'' prompt.

Semantic proximity between narratives and speeches was quantified using cosine similarity. Similarity scores were averaged across narratives within each supernarrative to capture high-level trends. To control for keyword-driven effects, we additionally constructed a control narrative at the presumbaly opposite end of the rhetorical spectrum (``We need to respect humanitarian principles in handling migration'').

To validate that narrative similarity captured meaningful political differences, we correlated similarity scores with expert-coded party positions on migration policy and overall ideology \citep{rovny_CHES_2024}. Pearson correlations were evaluated using Bonferroni-adjusted significance thresholds. Temporal trends and party-block differences in narrative prevalence were analysed using linear mixed-effects models with random intercepts and slopes at the party-block level. Because separate models were estimated for each of the five supernarratives plus the control, significance levels were Bonferroni-corrected to $p = 0.05/6$.
    
\section{Results}\label{sec:results}

\begin{figure}[ht]
\vskip 0.2in
\centering
\centerline{\includegraphics[width=\linewidth]{"fig/fig3.pdf"}}
\caption{Political bloc embeddings in reduced PLS space over time. Shading denotes bootstrapped 95\% confidence intervals of mean yearly embeddings. }\label{fig:fig3_pls}
\vskip -0.2in
\end{figure}

\begin{figure*}[ht]
\vskip 0.2in
\begin{center}
\centerline{\includegraphics[width=\textwidth]{"fig/fig4_search.pdf"}}
\caption{Semantic similarity to two migration supernarratives and the control prompt: temporal trajectories and top correlations to expert party ratings (CHES scores). Shading denotes bootstrapped 95\% confidence intervals of mean similarity.}
\label{fig:fig4_search}
\end{center}
\vskip -0.2in
\end{figure*}


\subsubsection{Topic Modelling with LDA}

To isolate migration-related discourse from all speeches, we use \emph{Latent Dirichlet Allocation} \citep{blei2003latent}. LDA models each speech as a probabilistic mixture of a predefined number of topics, where each topic is defined by a distribution over words. For every speech, the model estimates the probability of belonging to each topic.

We evaluated multiple model specifications and selected the model based on topic coherence (final score: 0.56) and manual inspection of topic interpretability. The selected model comprises 30 topics, one of which assigned highest probabilities to the words \emph{refugee}, \emph{border}, and \emph{migration}. Speeches were categorized as migration-related if they had an above-threshold probability for this topic ($n = 9705$).

The threshold was determined by two raters sampling 100 speeches from the probability range where the cutoff was expected based on initial tests %([0.20, 0.35])
and classifying whether they were migration-related. Using receiver operating characteristic analysis, we identified the threshold that minimized the difference between true and false positive rates. %(prob = 0.25).

% Why LDA and not something else?
% (TODO why LDA?)
% Description LDA
% LDA is a probabilistic topic model. 
% It assumes that in the analyzed corpus (here: the collection of speeches) there is a set of topics, which are probability distributions over all words in the corpus. 
% It considers each document (here: a speech) as a bag of words that were sampled from these topics. 
% For example, if a topic had high probabilities for the words "fish", "net", "water", then documents covering "fishing" would (under LDA's assumptions) have a high probability of being labelled as that topic. 
% Our LDA


% We tested different parameters (numer of topics, number of iterations over the dataset) and compared 
% the resulting topic coherence (TODO explanation) and the fidelity of the topics through manual inspection.
% Our final model contains 30 topics (for 10 iterations) --- one of which assigns highest probabilities to the words X, Y, Z (TODO words), which we call "migration topic".

% threshold


\subsection{Dimensions separating party blocs}\label{PLSresults}

PLS identifies directions in the embedding space that separate the party blocs meaningfully, with an averaged, cross-validated F1 score of $0.45 \pm 0.35$ (mean $ \pm$ std.\@ across validation folds). The most extreme examples of the corpus vocabulary, as embedded and reduced to two-dimensional space, are displayed in \Cref{fig:fig3_pls} and used for axis interpretation (see \Cref{sec:conclusion}).


\begin{figure*}[ht]
\vskip 0.2in
\begin{center}
\centerline{\includegraphics[width=\textwidth]{"fig/fig4_search.pdf"}}
\caption{Temporal trajectories of semantic similarity to selected migration narratives for different party blocks. Party ratings taken from the Chapel Hill Expert survey. Shadings indicate bootstrapped 95\% confidence intervals.}
\label{fig:fig4_search}
\end{center}
\vskip -0.2in
\end{figure*}

\subsection{Similarities to Established Migration Narratives}\label{SemSearchResults}

Mean semantic similarities of speeches to two exemplary anti-immigration supernarratives and our constructed humanitarian prompt are visualized in \Cref{fig:fig4_search}. Mixed linear models revealed party differences for the three anti-immigration supernarratives ``immigration is a threat'', ``immigrants' culture is problematic', and ``immigration is a burden'', which were significantly higher for far-right speakers (all $d_{\cos} = 0.27$), compared to all other blocks (for threat \& problematic $d_{\cos} = 0.25$, for burden $d_{\cos} = 0.24$, all $p < .003$). For narratives falling under ``immigrants as victims'' or the populist ``Us vs. Them'', and the comparison ``Humanitarian'' prompt, no consistent differences emerged. For all supernarratives, no significant temporal trends were found.

Out of all expert party ratings, similarities to the three anti-immigration tropes were most correlated with anti-Islam rhetoric ($r = [.39,.45]$) and salience of immigration in their political agenda ($r = [.37,.40]$), on third position followed salience of multiculturalism ($r_{threat} = .37, r_{burden} = .39$) and populist people vs.\ elite ($r_{problem} = .36$). Immigrants as victims framing had lower correlations with similar dimensions, the us vs.\ them narrative category correlated most with its equivalent people vs.\ elite rating ($r = .35$). The humanitarian comparison narrative yielded no significant CHES score correlations.



\section{Discussion \& Conclusion}\label{sec:conclusion}

% Use this section to briefly summarize the entire text. Highlight limitations and problems, but also make clear statements where they are possible and supported by the analysis.

Our analysis of European Parliament speeches (2014--2024) reveal systematic differences in migration rhetoric between party groups. 

% LDA & Agenda setting  
% TODO: talk about LDA misclassification rate as limitations? 

Far-right parties indicate active agenda-setting behavior by introducing migration into debates, particularly about economy and security. For other party blocs, this phenomenon is too rare to assert a defnite pattern and can partly be drawn to misclassifications of the LDA model. However, there are indications of a polar divide: Green and left parties tend to introduce migration to debates on humanitarian issues. Examining whether such a left-right distinction shapes where migration discourse is fed into the debate is a promising direction for further research.

PLS analysis, while constrained by noisy, unbalanced parliamentary data and the use of general-purpose embeddings, is a useful exploratory tool for uncovering latent political discourse dimensions in high-dimensional textual embeddings,
although it is vital to ground such finding in expert domain knowledge.

Semantic similarity results indicate that far-right rhetoric in the EP is characterized by a higher adoption of previously identified anti-immigrant narratives, without evidence of growth over the last decade. High correlations with established political scales support the validity of this approach as a computational tool for quantifying rhetoric tropes. At the same time, the embedding approach to political speech analysis cannot fully distinguish between mere invocation of statements and actual evaluative stances.

Across all analyses, right-wing populist speech emerges as distinct from other parties. In particular, parties on the political right appear more likely to foreground migration-related themes, while placing less emphasis on humanitarian concerns and framing immigrants in terms of threat, burden, and illegality.


% TODO: einräumen, dass semantic search aber auch keyword search related sein könnte und deswegen fehlgeleitet sein könnte 

% Methodologically, we demonstrate how combining LDA topic modeling, semantic embeddings, and narrative detection can capture both topical focus and rhetorical framing. While translation consistency and temporal confounding present limitations, our findings illuminate how populist discourse enters mainstream parliamentary institutions, with implications for European democratic deliberation.
% \section*{Notes}

% Your entire report has a \textbf{hard page limit of 4 pages} excluding references and the contribution statement. (I.e. any pages beyond page 4 must only contain the contribution statement and references). Appendices are \emph{not} possible. But you can put additional material, like interactive visualizations or videos, on a githunb repo (use \href{https://github.com/pnkraemer/tueplots}{links} in your pdf to refer to them). Each report has to contain \textbf{at least three plots or visualizations}, and \textbf{cite at least two references}. More details about how to prepare the report, inclucing how to produce plots, cite correctly, and how to ideally structure your github repo, will be discussed in the lecture, where a rubric for the evaluation will also be provided.


\newpage
\section*{Contribution Statement}

Giorgi Gogelashvili worked on data cleaning, analyzing and selecting the embedding model, and dimensionality reduction.

Samia Haque conducted preliminary data analysis, literature review and report editing. 

Jakob Kleine implemented speech translation, LDA, analyzed agenda setting, and organized the repository. 

Dennis Stroh prepared and merged the Parllaw and CHES datasets, and worked on data cleaning.

Quirin Unterguggenberger co-evaluated and visualized LDA, analyzed narrative similarities, and wrote the majority of the abstract and introduction.

All authors conducted initial literature review and jointly wrote the report. 

\bibliography{bibliography}
\bibliographystyle{icml2025}

\end{document}


% This document was modified from the files available at https://icml.cc/Conferences/2025/AuthorInstructions
% the full copyright notice is available within the file icml2025.sty
