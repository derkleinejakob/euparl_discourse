\section{Introduction}\label{sec:intro}

The continued success of right-wing populist parties in the 21st century is widely regarded as a major threat to European democracy and integration \citep{fossum_what_2023,rummens_populism_2017}. Populist rhetoric is commonly defined as constructing an antagonism between a ``pure people'' and a ``corrupt elite'' \citep{mudde_populist_2007}. Right-wing populism is closely tied to the issue of immigration: far-right parties have played a central role in its increasing politicisation \citep{hutter_politicising_2022}, which represents a crucial factor for their political success \citep{kende_xenophobia_2020}.

Over the past decade, immigration has become an increasingly salient issue in European election campaigns \citep{dekeyser_elections_2023} as well as in media coverage \citep{greussing_shifting_2017}, especially since the onset of the so-called ``refugee crisis'' in 2015 \citep{brueckner_crisis_2026}. Expanding this finding to parliamentary debates, we ask: \emph{(RQ1) How prevalent is migration in EP debates and do parties strategically put it on the agenda?}

Electoral gains of populist parties have affected the nature of parliamentary discourse \citep{schwalbach2023talking}. A recent quantitative analysis of EP speech embeddings has identified a gradual increase in emotional rhetoric from 1999 to 2022, with right-wing populist groups leading the trend \citep{subtil_verger_2024}. In the German national parliament, an LLM-based study has revealed increasing anti-solidarity messaging around immigration, not only for right-wing, but also christian-conservative and liberal parties \citep{kostikova_llm_2025}. Seeing these underlying shifts in rhetoric, we investigate: \emph{(RQ2) How does migration rhetoric differ between parties and how do these patterns evolve over time?}

More fine-grained analyses have revealed the use of common underlying narratives, defined as ``patterns of interpretation'' through which the issue is relayed to the public. Social media posts from populist leaders commonly employ anti-immigrant frames like ``immigrants take our jobs'' \citep{seiger_navigating_2025}. Consequently, we pose \emph{(RQ3): Do parties resort to known anti-immigration narratives?}

The investigation of these questions shed light on how the growing prominence of right-wing populism and immigration as a salient political issue manifest in EP debates, with potential implications for broader societal discourse and legislative outcomes. Analyzing all parliamentary speeches between 2014 and 2024, we employ unsupervised topic modeling to break down the prevalence of migration debates relative to other topics over the last decade, and provide evidence of agenda-setting strategies employed by right-wing populist groups (\Cref{sec:methods}). By embedding migration-labeled speeches into vector representations, we examine the semantic dimensions along which party groups can be differentiated and identify an increased use of anti-immigration narratives by right-wing groups (\Cref{sec:results}).

% previous versions:

% Building on these foundations, we apply computational methods to European Parliament debates to analyze migration discourse. We combine topic modeling to measure issue salience, semantic embeddings to map ideological positioning, and narrative detection to quantify the adoption of populist rhetorical frames across party groups.

% Speeches are first classified into topics using Latent Dirichlet Allocation (LDA) to identify immigration-related debate and to estimate its prominence over time. Analyses of the distribution of migration-related speeches across predefined debate agendas provide quantitative evidence consistent with agenda-setting strategies employed by right-wing populist groups. With the use of speech embeddings, we examine the semantic dimensions along which party groups can be differentiated and find evidence for an increased use of previously identified anti-immigration narratives by right-wing groups compared to moderate factions.

% instructions: 

% Motivate the problem, situation or topic you decided to work on. Describe why it matters (is it of societal, economic, scientific value?). Outline the rest of the paper (use references, e.g.~to \Cref{sec:methods}: What kind of data you are working with, how you analyse it, and what kind of conclusion you reached. The point of the introduction is to make the reader want to read the rest of the paper.