\section{Introduction}\label{sec:intro}

The continued success of right-wing populist parties in the 21st century is widely regarded as a major threat to European democracy and integration \citep{fossum_what_2023,rummens_populism_2017}. While populist rhetoric in general is characterized by a constructed antagonism between a ``pure people'' and a ``corrupt elite'' \citep{mudde_populist_2007}, right-wing populism in particular is closely tied to the issue of immigration: far-right parties have played a central role in its growing politicisation \citep{hutter_politicising_2022}, which represents a crucial factor of their political success \citep{kende_xenophobia_2020}.

Over the past decade, immigration has become an increasingly salient issue in European election campaigns \citep{dekeyser_elections_2023} and media coverage \citep{greussing_shifting_2017}, especially since the onset of the so-called ``refugee crisis'' in 2015 \citep{brueckner_crisis_2026}. In light of these trends, we ask $Q_1$\emph{: How prevalent is migration in EP debates and do parties strategically put it on the agenda?}

Electoral gains of populist parties have affected the nature of parliamentary discourse \citep{schwalbach2023talking}. A recent quantitative analysis of EP speech embeddings identified a gradual rise in emotional rhetoric from 1999 to 2022, with right-wing populist groups leading the trend \citep{subtil_verger_2024}. In the German national parliament, an LLM-based study found increasing anti-solidarity messaging around immigration, not only for right-wing, but also christian-conservative and liberal parties \citep{kostikova_llm_2025}. Seeing these underlying shifts in rhetoric, we investigate $Q_2$\emph{: How does migration rhetoric differ between parties and how do these patterns evolve over time?}

More fine-grained analyses revealed the use of common underlying narratives, defined as ``patterns of interpretation'' through which the issue is relayed to the public. Social media posts from populist leaders commonly employ anti-immigrant frames like ``immigrants take our jobs'' \citep{seiger_navigating_2025}. $Q_3$\emph{: Do parties resort to known anti-immigration narratives?}

Addressing these questions clarifies how the growing prominence of right-wing populism and immigration as a salient political issue manifest in EP debates. %, with potential implications for broader societal discourse and legislative outcomes.
Analyzing all parliamentary speeches between 2014 and 2024, we employ unsupervised topic modeling to break down the relative prevalence of migration discourse and provide evidence of agenda-setting strategies by right-wing populist groups (\Cref{LDAresults}). By embedding migration-related speeches, we examine semantic dimensions that differentiate party groups (\Cref{PLSresults}) and identify higher use of anti-immigration narratives among right-wing speakers (\Cref{SemSearchResults}).