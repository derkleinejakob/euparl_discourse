Motivated by the rise of populism in Europe since the late 1990s, this study investigates ideological shifts in European Parliament (EP) speeches using natural language processing. Drawing on the novel ParLawSpeech dataset \citep{SDN-10.7802-2824} which contains 574,199 speeches from 1999 to 2024 alongside metadata on speaker identity, we use sentence embedding models to examine the semantic content and emotional tone of parliamentary debates over time.

We expect that speech embeddings will form clusters reflecting party affiliation and ideological alignment. In step with recent political developments, we further hypothesize an increase in negative sentiment within the immigration debate among centrist and right-wing groups, accompanied by growing semantic similarity between these two factions over the past two decades. Finally, we test whether established migration-related narratives associated with right-wing populism can be identified in parliamentary discourse and how their prevalence has developed over time.
