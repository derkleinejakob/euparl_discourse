
% Use this section to briefly summarize the entire text. Highlight limitations and problems, but also make clear statements where they are possible and supported by the analysis.

Our analysis of European Parliament speeches (2014--2024) reveal systematic differences in migration rhetoric between party groups. 

% LDA & Agenda setting  
% TODO: talk about LDA misclassification rate as limitations? 

Far-right parties indicate active agenda-setting behavior by introducing migration into debates, particularly about economy and security. For other party blocs, this phenomenon is too rare to assert a defnite pattern and can partly be drawn to misclassifications of the LDA model. However, there are indications of a polar divide: Green and left parties tend to introduce migration to debates on humanitarian issues. Examining whether such a left-right distinction shapes where migration discourse is fed into the debate is a promising direction for further research.

PLS analysis, while constrained by noisy, unbalanced parliamentary data and the use of general-purpose embeddings, is a useful exploratory tool for uncovering latent political discourse dimensions in high-dimensional textual embeddings,
although it is vital to ground such finding in expert domain knowledge.

Semantic similarity results indicate that far-right rhetoric in the EP is characterized by a higher adoption of previously identified anti-immigrant narratives, without evidence of growth over the last decade. High correlations with established political scales support the validity of this approach as a computational tool for quantifying rhetoric tropes. At the same time, the embedding approach to political speech analysis cannot fully distinguish between mere invocation of statements and actual evaluative stances.

Across all analyses, right-wing populist speech emerges as distinct from other parties. In particular, parties on the political right appear more likely to foreground migration-related themes, while placing less emphasis on humanitarian concerns and framing immigrants in terms of threat, burden, and illegality.


% TODO: einräumen, dass semantic search aber auch keyword search related sein könnte und deswegen fehlgeleitet sein könnte 

% Methodologically, we demonstrate how combining LDA topic modeling, semantic embeddings, and narrative detection can capture both topical focus and rhetorical framing. While translation consistency and temporal confounding present limitations, our findings illuminate how populist discourse enters mainstream parliamentary institutions, with implications for European democratic deliberation.
% \section*{Notes}

% Your entire report has a \textbf{hard page limit of 4 pages} excluding references and the contribution statement. (I.e. any pages beyond page 4 must only contain the contribution statement and references). Appendices are \emph{not} possible. But you can put additional material, like interactive visualizations or videos, on a githunb repo (use \href{https://github.com/pnkraemer/tueplots}{links} in your pdf to refer to them). Each report has to contain \textbf{at least three plots or visualizations}, and \textbf{cite at least two references}. More details about how to prepare the report, inclucing how to produce plots, cite correctly, and how to ideally structure your github repo, will be discussed in the lecture, where a rubric for the evaluation will also be provided.
