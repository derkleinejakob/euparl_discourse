
% Use this section to briefly summarize the entire text. Highlight limitations and problems, but also make clear statements where they are possible and supported by the analysis.

% Our analysis of European Parliament speeches (2014--2024) reveals systematic patterns in migration discourse. Topic modeling shows right-wing parties disproportionately drive migration salience while strategically introducing it into unrelated debates—particularly economic and security discussions—demonstrating active agenda-setting behavior. Partial Least Squares analysis indicates semantic convergence between centrist and right-wing rhetoric following the 2015 refugee crisis, suggesting diffusion of populist framing. 

[TODO: summary sentences for each analysis, limitations, conclusion]

While analysis is constrained by noisy, unbalanced parliamentary data and the use of general-purpose embeddings, the results suggest that PLS could serve as a useful exploratory tool for uncovering latent political discourse dimensions in high-dimensional textual embeddings,
although it is vital to ground such finding in expert domain knowledge.

% semantic search
Our results indicate that far-right rhetoric in the EP is characterized by a higher adoption of previously identified anti-immigrant narratives, without evidence of growth over the last decade. High correlations with established political scales indicate that this approach is a viable and promising computational tool for quantifying rhetoric tropes.  

% Methodologically, we demonstrate how combining LDA topic modeling, semantic embeddings, and narrative detection can capture both topical focus and rhetorical framing. While translation consistency and temporal confounding present limitations, our findings illuminate how populist discourse enters mainstream parliamentary institutions, with implications for European democratic deliberation.
% \newpage

% \section*{Notes}

% Your entire report has a \textbf{hard page limit of 4 pages} excluding references and the contribution statement. (I.e. any pages beyond page 4 must only contain the contribution statement and references). Appendices are \emph{not} possible. But you can put additional material, like interactive visualizations or videos, on a githunb repo (use \href{https://github.com/pnkraemer/tueplots}{links} in your pdf to refer to them). Each report has to contain \textbf{at least three plots or visualizations}, and \textbf{cite at least two references}. More details about how to prepare the report, inclucing how to produce plots, cite correctly, and how to ideally structure your github repo, will be discussed in the lecture, where a rubric for the evaluation will also be provided.
