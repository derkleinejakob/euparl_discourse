
Our analysis of European Parliament speeches (2014 to 2024) reveal systematic differences in migration rhetoric.

Far-right parties appear to employ agenda-setting behavior by introducing migration-speech into debates, particularly about the economy and security. 
For other party blocs, this phenomenon is too rare to assert a definite pattern and could partly represent LDA misclassifications. 

Dimensionality reduction reveals two axes that separate parties in semantic space. The first axis can be interpreted as \emph{conciliatory} $\Leftrightarrow$ \emph{oppositional} mode of migration discourse. The second axis is less distinct and is driven by extreme cases referencing security-related concerns, particularly in an Eastern European context, suggesting a \emph{moral} $\Leftrightarrow$ \emph{pragmatic-benefits / security} dimension. Despite noise in general-purpose embeddings, party blocs separate consistently along both axes, indicating that PLS can recover meaningful latent political dimensions.

Semantic similarity results indicate that far-right rhetoric in the EP is characterized by a higher adoption of previously identified anti-immigrant narratives, without evidence of growth over the last decade. High correlations with established political scales support the validity of this approach as a computational tool for quantifying rhetorical tropes. At the same time, the embedding approach to political speech analysis cannot fully distinguish between mere invocation of statements and actual evaluative stances.

Across all analyses, right-wing populist speech emerges as distinct from other parties. In particular, parties on the political right appear more likely to foreground migration-related themes, while placing less emphasis on humanitarian concerns and framing immigrants in terms of threat, burden, and illegality.
