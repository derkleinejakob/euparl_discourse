
% Use this section to briefly summarize the entire text. Highlight limitations and problems, but also make clear statements where they are possible and supported by the analysis.

Our analysis of European Parliament speeches (2014--2024) reveals systematic patterns in migration discourse. 
% Topic modeling shows right-wing parties disproportionately drive migration salience while strategically introducing it into unrelated debates—particularly economic and security discussions—demonstrating active agenda-setting behavior. Partial Least Squares analysis indicates semantic convergence between centrist and right-wing rhetoric following the 2015 refugee crisis, suggesting diffusion of populist framing. 



% LDA & Agenda setting  
% TODO: talk about LDA misclassification rate as limitations? 
Topic modeling shows a continuous prevalence of migration-related debate since 2014. Far-right parties indicate active agenda-setting behavior by introducing migration into debates, particularly about economy and security. For other party blocks, this phenomenon is too rare to assert a defnite pattern. However, there are indications of a polar divide: green and left parties tend to link migration to debates on humanitarian issues. Examining whether such a left-right distinction shapes where migration discourse is fed into the debate is a promising direction for further research.


% shortcoming of the analysis: does not allow for analyizing polar-besetzungen von migrations-verknpüfung    

% semantic search
Our results indicate that far-right rhetoric in the EP is characterized by a higher adoption of previously identified anti-immigrant narratives, without evidence of growth over the last decade. High correlations with established political scales indicate that this approach is a viable and promising computational tool for quantifying rhetoric tropes.  % TODO: einräumen, dass semantic search aber auch keyword search related sein könnte und deswegen fehlgeleitet sein könnte 


% Methodologically, we demonstrate how combining LDA topic modeling, semantic embeddings, and narrative detection can capture both topical focus and rhetorical framing. While translation consistency and temporal confounding present limitations, our findings illuminate how populist discourse enters mainstream parliamentary institutions, with implications for European democratic deliberation.
% \newpage

% \section*{Contribution Statement}
% Explain here, in one sentence per person, what each group member contributed. For example, you could write: Max Mustermann collected and prepared data. Gabi Musterfrau and John Doe performed the data analysis. Jane Doe produced visualizations. All authors will jointly wrote the text of the report. Note that you, as a group, a collectively responsible for the report. Your contributions should be roughly equal in amount and difficulty.

% \section*{Notes}

% Your entire report has a \textbf{hard page limit of 4 pages} excluding references and the contribution statement. (I.e. any pages beyond page 4 must only contain the contribution statement and references). Appendices are \emph{not} possible. But you can put additional material, like interactive visualizations or videos, on a githunb repo (use \href{https://github.com/pnkraemer/tueplots}{links} in your pdf to refer to them). Each report has to contain \textbf{at least three plots or visualizations}, and \textbf{cite at least two references}. More details about how to prepare the report, inclucing how to produce plots, cite correctly, and how to ideally structure your github repo, will be discussed in the lecture, where a rubric for the evaluation will also be provided.
