
\subsubsection{Topic Modelling with LDA}
To identify how parties talk about migration, we first have to assign speeches to a semantic topic.
For this, we use Latent Dirichlet Allocation \emph{(LDA)} [TODO: source]. 
% What is LDA 
LDA fits a probabilistic model on a corpus of documents (here: our parliamentary speeches) and identifies a fixed number of topics in the corpus which are characterized by words that have a high probability of appearing in the topic. For each speech and topic, it assigns a probability of that speech belonging to the topic. 

% Why LDA and not something else?
% (TODO why LDA?)
% Description LDA
% LDA is a probabilistic topic model. 
% It assumes that in the analyzed corpus (here: the collection of speeches) there is a set of topics, which are probability distributions over all words in the corpus. 
% It considers each document (here: a speech) as a bag of words that were sampled from these topics. 
% For example, if a topic had high probabilities for the words "fish", "net", "water", then documents covering "fishing" would (under LDA's assumptions) have a high probability of being labelled as that topic. 
% Our LDA

We compared different model fits and chose one based on the its coherence score (final coherence: XX) and throguh manually inspecting the fidelity of the created topics. Our final model contains 30 topics, one of which assigned highest probabilities for the words X, Y, Z (TODO words), which we call "migration topic".

% We tested different parameters (numer of topics, number of iterations over the dataset) and compared 
% the resulting topic coherence (TODO explanation) and the fidelity of the topics through manual inspection.
% Our final model contains 30 topics (for 10 iterations) --- one of which assigns highest probabilities to the words X, Y, Z (TODO words), which we call "migration topic".

% threshold
To determine an appropriate migration-topic probability threshold, two raters sampled 100 speeches from the probability range where the cutoff was expected based on initial tests (prob = [0.20, 0.35]) and classified whether they were migration-related. We then used receiver operating characteristic (ROC) analysis to identify the threshold that minimized the difference between true and false positive rates (prob = 0.25).

(TODO finish this)