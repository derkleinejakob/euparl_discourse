The continued success of right-wing populist parties in the 21st century is widely regarded as a major threat to European democracy and integration \citep{fossum_what_2023,rummens_populism_2017}. Populist rhetoric is commonly defined as constructing an antagonism between a `pure people' and a `corrupt elite' \citep{mudde_populist_2007}. Right-wing populism is also closely tied to the issue of immigration. Parties of this ideology have played a central role in the increasing politicisation of immigration \citep{hutter_politicising_2022}, which represents a crucial factor for their political success \citep{kende_xenophobia_2020}. Over the past decade, immigration has become an increasingly salient issue in European election campaigns \citep{dekeyser_elections_2023} as well as in media coverage \citep{greussing_shifting_2017}.

Electoral gains of populist parties have manifested in significant changes of parliamentary discourse \citep{schwalbach2023talking}. A recent quantitative analysis of EP speech embeddings has identified a gradual increase in emotional rhetoric since 1999, with right-wing populist groups leading the trend \citep{subtil_verger_2024}. In the German national parliament, an LLM-based study has revealed increasing anti-solidarity messaging around immigration, not only for right-wing, but also christian-conservative and liberal parties \citep{kostikova_llm_2025}. This trend begins around 2015, which marks the onset of the so-called `refugee crisis' \citep{brueckner_crisis_2026}.

More fine-grained analyses of the migration discourse have revealed the use of common underlying narratives, defined as `selective depictions of reality' and `patterns of interpretation' through which the issue is relayed to the public. Social media posts from populist leaders commonly employ anti-immigrant frames like `immigrants take our jobs' or anti-establishment narratives such as `our sovereignty is under threat' \citep{seiger_navigating_2025}.

% Building on these foundations, we apply computational methods to European Parliament debates to analyze migration discourse. We combine topic modeling to measure issue salience, semantic embeddings to map ideological positioning, and narrative detection to quantify the adoption of populist rhetorical frames across party groups.
% this feels redundant with the next two paragraphs


% agenda setting!

This report provides a quantitative assessment of how the growing prominence of right-wing populism and immigration as a salient political issue manifests in debates in the European Parliament, with potential implications for broader societal discourse and legislative outcomes. All parliamentary speeches between 2004 and 2024 as recorded by the ParlLawSpeech dataset inform the analyses. 

Speeches are first classified into topics using Latent Dirichlet Allocation (LDA) to identify immigration-related debate and to estimate its prominence over time. Analyses of the distribution of migration-related speeches across predefined debate agendas provide quantitative evidence consistent with agenda-setting strategies employed by right-wing populist groups. With the use of speech embeddings, we examine the semantic dimensions along which party groups can be differentiated and find evidence for an increased use of previously identified anti-immigration narratives by right-wing groups compared to moderate factions.

% Motivate the problem, situation or topic you decided to work on. Describe why it matters (is it of societal, economic, scientific value?). Outline the rest of the paper (use references, e.g.~to \Cref{sec:methods}: What kind of data you are working with, how you analyse it, and what kind of conclusion you reached. The point of the introduction is to make the reader want to read the rest of the paper.
