% quirin's shortening proposal:
% NOTE by Jakob! please dont abbreviate ParlLawSpeech with PLS, we use PLS to abbreviate Partial Least Squares!!

We use the ParlLawSpeech dataset \citep{SDN-10.7802-2824}, which contains the full verbatim transcriptions of parliamentary debates between 1999 and 2024 with rich metadata. We restrict analyses to last two complete legislative periods (2014–2024), as the 2015/2016 “refugee crisis” could mark a qualitative shift in the nature and salience of migration debates that might bias topic-model estimation. In a preliminary test, evidence for this effect was reflected by an increase in migration-labeled speeches from around 1\% before 2014 to over 2.5\% thereafter.

To contextualize and validate our findings around rhetorical patterns, we draw on expert-coded party positions from the Chapel Hill Expert Survey (CHES; \citealp{rovny_CHES_2024}), which provides repeated cross-sectional estimates of party ideology (e.g.\ left-right spectrum) and policy positions (e.g.\ on immigration) using standardized scales. CHES items were rescaled to a common 0–10 metric where necessary, following \citet{adams2014voters}. To link the CHES ratings to our speech data, we enriched the corpus with speaker-level metadata from the EP’s Open Data Portal, including national party affiliation at the time of speech delivery. 

While parliamentary speeches capture how parties frame political issues, CHES offers an independent, expert-based assessment of ideological positioning, allowing us to validate patterns in political discourse against an external benchmark. We use ratings from the 2014, 2019, and 2024 surveys and compute correlations with semantic speech measures from the corresponding years.

% Dennis' original version
% To analyze migration discourse in the EP, we rely on the \textit{ParlLawSpeech} (PLS) dataset \citep{SDN-10.7802-2824}, providing us with a large-scale corpus of more than 500,000
% verbatim parliamentary speeches alongside rich metadata, covering European legislatures between 1999 and 2024.
% We restrict our analyses to the last two complete legislative periods (2014–2024), as the 2015/2016 “refugee crisis” marks a qualitative shift in nature and salience of 
% migration-related debates that could bias topic-model estimation. This shift is reflected in the proportion of migration-labeled speeches, which remains around 1\% prior to 2014 but rises to over 2.5\% thereafter.
% We further enrich the corpus with speaker-level metadata obtained from the EP’s Open Data Portal API, i.e. national party affiliation of 
% each Member of Parliament at the time of speech delivery. To contextualize and validate patterns observed in parliamentary speech, we also draw on expert's perceptions of party positions from the \textit{Chapel Hill Expert Survey} 
% (CHES) of \citet{rovny_CHES_2024}. CHES is a long-running study, surveying hundreds of country experts to estimate the ideological positions of national political parties across Europe on a 
% wide range of dimensions, such as general left–right ideology, European integration, and issue-specific policy stances, e.g. experts are asked to place parties on scales from 0 ("strongly favors liberal immigration policy")
% to 10 ("strongly favors restrictive policy").
% Most CHES score dimensions range from 0-10, except for a few survey items (e.g. having a scale of 1-7) which we recalibrated to match the 0-10 range, as done in \citep{adams2014voters}.
% Since 1999, the survey has been conducted roughly every four years, including 2014, 2019 and 2024, offering repeated cross-sectional measurements of 
% party positions and enabling the study of ideological change over time.
% While parliamentary speeches capture how parties communicate and frame political issues, CHES provides an independent, expert-based assessment of where parties are 
% positioned in ideological space. Combining the two allows us to triangulate latent patterns in political discourse against an external benchmark that is not derived 
% from the same textual data.
% Exluded:
%Importantly, CHES scores are not used to label or classify individual speeches, but instead serve as contextual variables against which aggregated 
%speech patterns can be compared.

In 2015 and 2016, the total number of speeches increases sharply (over 50,000 per year), compared to an average of 10,098 in other years (\textit{SD} = 2,724), reflecting 
a shift from numerous, shorter to fewer and longer speeches over time.
This phenomenon is largely explained by changes in parliamentary procedure adopted around the end of 2016 \citep{tremblay2024open}, most notably the discontinuation of short written statements expressing opinions on specific issues. Their removal largely normalizes annual speech counts. For the main analyses, we retain written declarations, as they often contain substantive party positions. Because such declarations could be co-signed by multiple Members and thus appear duplicated in the dataset, we retain only the first instance.

Due to the fragmentation of right-wing forces in the EP and realignment across legislative terms, parliamentary groups were sorted into six ideological blocks following \citet{kaiser_seventy_2023}, see \Cref{fig:fig1_lda}.

% TODO specific numbers/facts on CHES<->PLS merging? (how many rows enriched etc.)
% By employing Latent Dirichlet Allocation (TODO verweis sec), we identify speeches related to the topic of migration, and after only including the legislative periods between 2014 and 2024
% this leaves us with 9,705 datapoints in total.
% After only including speeches of legislative terms between 2014 and 2024, this leaves us with XXXX datapoints in total.
% -> put this at the end of the preprocessing section?
%
%%% OLD TEXT
%We are using the novel \textit{ParlLawSpeech} (PLS) dataset from Schwalbach et al. 2025 for the investigation of our study. 
%It contains more than 570,000 plenary speeches from legislative periods of the European parliament (EP) between 1999 and 2024.
%The authors also provide (partially) machine translated text in English for about 40\% of the speeches, since the EP stopped providing official translations around the end of 2012. 
%Furthermore, the dataset contains metadata on the speakers and the speeches given,
%e.g. date and agenda item under which the speech was given, if submission was in written form and/or from multiple \textit{members of parliament} (MEPs), or the speaker's 
%party affiliation (referring to European political parties/groups), among other. We further enriched
%the dataset with metadata accessible from the public API of the EP's "Open Data Portal", in particular the national party affiliations of each speaker (by using the \textit{EP-ID} of the 
%respective MEPs). This allowed us to link the PLS dataset with the \textit{Chapel Hill Expert Survey} (CHES)
%from Rovny, Bakker et al. 2025. The CHES dataset estimates party positioning on European integration, ideology (e.g. left/right) and policy issues for national parties in all member states of 
%the European Union (EU). The study surveyed hundreds of experts roughly every four years
%between 1999 and 2024 and more recently (**TODO**: since when???) also includes ratings of non-EU policy issues such as immigration or anti-elite rhetoric (**TODO:** which are relevant in 
%particular?) Assuming that the ideological orientation of a speaker's affiliated national party
%roughly reflects his own position, the CHES data set could help us to better control our analyses, as membership of a European party (group) presumably allows for less detailed/granular 
%statements/assumptions.
%We restrict our analyses to the last two complete legislative periods (2014–2024), as the 2015/2016 “refugee crisis” marks a qualitative shift in the nature and salience of 
%migration-related debates that could bias topic-model estimation. This shift is reflected in the proportion of migration-labeled speeches, which remains around 1\% prior to 2014 but rises to over 2.5\% thereafter.
%
%% OLD TEXT: 2015 and 2016 anomaly 
%\textbf{2015 and 2016 anomaly} In 2015 and 2016, there is a drastic increase in the number of speeches compared to the other years (72,964 in 2015 and 16 on average, compared to 10,098 in the other years on average). 
%This difference can be explained with a rule-change that was adopted by the parliament at the end of 2016, discontinuing so-called 'written declarations' that allowed party members to hand in short expressions of 
%opinion on a certain issue [TODO ref]. Omitting all 'written' speeches (TODO: explain what written flag means) can normalize the number of speeches per year. However, for the general analysis, we keep written declarations as 
%part of the dataset because they can contain relevant stances of the parties on political issues. Since written declarations could be co-signed by multiple speakers, they can appear duplicated in the dataset. 
%These duplicated items were removed, keeping only the first instance of a duplicated speech.
