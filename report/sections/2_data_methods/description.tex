We are using the novel \textit{ParlLawSpeech} (PLS) dataset from Schwalbach et al. 2025 for the investigation of our study. It contains more than 570,000 plenary speeches from legislative periods of the European parliament (EP) between 1999 and 2024.
The authors also provide (partially) machine translated text in English for about 40\% of the speeches, since the EP stopped providing official translations around the end of 2012. Furthermore, the dataset contains metadata on the speakers and the speeches given,
e.g. date and agenda item under which the speech was given, if submission was in written form and/or from multiple \textit{members of parliament} (MEPs), or the speaker's party affiliation (referring to European political parties/groups), among other. We further enriched
the dataset with metadata accessible from the public API of the EP's "Open Data Portal", in particular the national party affiliations of each speaker (by using the \textit{EP-ID} of the respective MEPs). This allowed us to link the PLS dataset with the \textit{Chapel Hill Expert Survey} (CHES)
from Rovny, Bakker et al. 2025. The CHES dataset estimates party positioning on European integration, ideology (e.g. left/right) and policy issues for national parties in all member states of the European Union (EU). The study surveyed hundreds of experts roughly every four years
between 1999 and 2024 and more recently (**TODO**: since when???) also includes ratings of non-EU policy issues such as immigration or anti-elite rhetoric (**TODO:** which are relevant in particular?) Assuming that the ideological orientation of a speaker's affiliated national party
roughly reflects his own position, the CHES data set could help us to better control our analyses, as membership of a European party (group) presumably allows for less detailed/granular statements/assumptions.


% 2015 and 2016 anomaly 
\textbf{2015 and 2016 anomaly} In 2015 and 2016, there is a drastic increase in the number of speeches compared to the other years (72,964 in 2015 and 16 on average, compared to 10,098 in the other years on average). This difference can be explained with a rule-change that was adopted by the parliament at the end of 2016, discontinuing so-called 'written declarations' that allowed party members to hand in short declarations on a certain issue. 

