\subsection{Interpreting semantic differences}\label{PLSresults}
\begin{figure}[ht]
\vskip 0.2in
\centering
\centerline{\includegraphics[width=\linewidth]{"fig/fig3.pdf"}}
\caption{Temporal movement of political block embeddings in reduced PLS space. Shading denotes bootstrapped 95\% confidence intervals of mean yearly embeddings. }\label{fig:fig3_pls}
\vskip -0.2in
\end{figure}

    The resulting micro-averaged F1 score from leave-one-out cross-validation is $ 0.45 \pm 0.35$ (mean $ \pm$ std. accross validation folds).
    While a clear interpretation of the underlying political dimensions requires substantial domain knowledge, we believe that combining word associations with extreme examples of speeches along each cardinal direction provides some clues about their connotations.
    Based on this analysis, we interpret the first PLS axis as a \textbf{conciliatory $\Leftrightarrow$  oppositional} discourse spectrum, although it's harder to discern underlying political dimension for the second axis, we presume it to be \textbf{moral / human-rights $\Leftrightarrow$ pragmatic-benefits / security} debate \autoref{fig:fig3_pls}. 
    In this discovered space Pearson correlation with CHES party ratings, shows that axis 2 is most strongly correlated with EU asylum ($r = -0.45$), anti-Islam rhetoric ($r = 0.46$), and attitudes toward ethnic minorities ($r = 0.44$), while Axis 1 displays the strongest correlations with protectionism ($r = 0.57$), EU asylum ($r = -0.55$), and people-versus-elite rhetoric ($r = 0.55$).