\subsection{Interpreting semantic differences}
\begin{figure*}[ht]
\vskip 0.2in
\centering
\centerline{\includegraphics[width=\linewidth]{"fig/fig3.pdf"}}
\caption{ \textbf{Left.} Position of each political group  \textbf{Right.} Movement of political groups over the time displayed separately for each dimension  }\label{fig:fig3_pls}
\vskip -0.2in
\end{figure*}

    [Should this be in Discussion???]While a clear interpretation of the underlying political dimensions requires substantial domain knowledge, we believe that combining word associations with extreme examples of speeches along each cardinal direction provides strong clues about their connotations.
    Based on this analysis, we interpret the first PLS axis as a \textbf{conciliatory $\Leftrightarrow$  oppositional} discourse spectrum, and the second axis as a \textbf{moral / human-rights $\Leftrightarrow$ pragmatic-benefits} debate \autoref{fig:fig3_pls}.

    Moral outrage and discussion of human rights violations have been consistently key aspects of both green-left blocks and parts of the right block. Along the first axis, we observe little to no movement over the years overall, suggesting that political blocks have largely maintained their characteristic way of conducting discourse. Nevertheless, there is a clear division between centrist and oppositional blocks, with greens often positioned in between. Oppositional blocks exhibit adversarial framing and conflict-driven rhetoric, whereas centrist blocks focus more on consensus-building.
    On the second axis, we observe a clear shift along the ethical–pragmatic spectrum. Between 2016 and 2020, many parties move from pragmatic policy framing towards more moral debates. Christian conservative and right-wing blocks remain closer to the axis center, while green and left blocks maintain stronger positions on the moral end of the spectrum.
