\subsection{Dimensions seperating party blocs}\label{PLSresults}

PLS identifies directions in the embedding space that seperate the party blocs meaningfully, with a micro-averaged, cross-validated F1 score of $0.45 \pm 0.35$ (mean $ \pm$ std. accross validation folds). To interpret the identified axes, embeddings of the full vocabulary were reduced to the two-dimensional space, and the most extreme per axis were selected as labels (Figure \ref{fig:fig3_pls}).

% old version: 
% The resulting micro-averaged F1 score from leave-one-out cross-validation is $ 0.45 \pm 0.35$ (mean $ \pm$ std. accross validation folds).
% While a clear interpretation of the underlying political dimensions requires substantial domain knowledge, combining word associations with extreme examples of speeches along each axis provides some clues.
% Based on this analysis, we interpret the first PLS axis as a \textbf{conciliatory $\Leftrightarrow$  oppositional} discourse spectrum. The second axis is less distinct, but likely reflects a \textbf{moral / human-rights $\Leftrightarrow$ pragmatic-benefits / security} dimension \autoref{fig:fig3_pls}. 
% Consistent with the interpretation, Pearson correlation with CHES party ratings shows that axis 2 correlates most strongly with EU asylum ($r = -0.45$), anti-Islam rhetoric ($r = 0.46$), and attitudes toward ethnic minorities ($r = 0.44$), while Axis 1 displays the strongest correlations with protectionism ($r = 0.57$), EU asylum ($r = -0.55$), and people-versus-elite rhetoric ($r = 0.55$).