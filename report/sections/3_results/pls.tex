\subsection{Interpreting semantic differences}
\begin{figure*}[ht]
\vskip 0.2in
\centering
\centerline{\includegraphics[width=\linewidth]{"fig/fig3.pdf"}}
\caption{ \textbf{Left:} Position of each political group. \textbf{Right:} Movement of political groups over the time displayed separately for each dimension. 
  }\label{fig:fig3_pls}
\vskip -0.2in
\end{figure*}

    To mitigate and quantify temporal bias in our PLS analysis, we excluded written speeches predominantly occurring in 2015 and 2016 and fitted the PLS model on the remaining 5,433 observations. Furthermore, we employed a leave-one-out cross-validation strategy adapted for temporal data \citep{https://doi.org/10.1111/ecog.02881} The resulting micro-averaged F1 score was $ 0.45 \pm 0.35$ (mean $ \pm$ std. accross validation folds).
    While a clear interpretation of the underlying political dimensions requires substantial domain knowledge, we believe that combining word associations with extreme examples of speeches along each cardinal direction provides some clues about their connotations.
    Based on this analysis, we interpret the first PLS axis as a \textbf{conciliatory $\Leftrightarrow$  oppositional} discourse spectrum, although it's harder to discern underlying political dimension for the second axis, we presume it to be \textbf{moral / human-rights $\Leftrightarrow$ pragmatic-benefits} debate \autoref{fig:fig3_pls}. [TODO CHESS correlations]

   