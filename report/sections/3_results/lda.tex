\subsection{Prevalence of migration speeches}

LDA organized all speeches into 30 interpretable and broadly coherent topics, with temporal trends in prevalence roughly consistent with salient world events, such as the outbreak of the Covid-19 pandemic in 2020 or the Russian invasion of Ukraine in 2022 (\Cref{fig:fig1_lda}). Although the proportion of migration-related speeches per year remained relatively stable, absolute counts peaked in 2015 and 2016, largely due to the use of written declarations (see \Cref{sec:methods}). [TODO: insert ratios per party here?]

\subsection{Setting the agenda}
% Agenda setting: 
% - control of agenda => political power (ref. to oxford paper?)
% - finding in germany: political right has influence over what other parties talk about. 
A phenomenon studied in the literature is \emph{agenda setting}, which describes a political acteur's strive to place their topic of interest in the discourse [TODO: ref]. In the context of parliamentary debate in the EU, it is interesting to ask: do parties try to put migration on the agenda? For this purpose, we looked at debates where all speeches that were labelled as migration-related stem from the same party block, indicating that migration is not the main topic of the debate, but continuously mentioned by that party block. We defined such a debate as an instance of agenda-setting by the respective party. All parties exhibit agenda-setting behaviour. 

However, it is drastically more pronounced for far-right parties, with X\% of their overall migration-related speech occuring debates where they are the only party talking about migration --- compared to Y\% for other parties on average [TODO: add significance statement]. Do parties systematically bring up migration in certain contexts? To answer this, we assigned each debate a 'true' topic, which is the topic with highest average probability among all speeches in that debate. Far-right parties bring up migration mostly in debates where the 'true' topic is related to foreign relations ('Trade Relations', 'EU Security / Defense', or 'Economic Development') while progressive parties (social democratic, green, left) link it most to humanitarian situations ('Disasters / Epidemics', 'International Conflicts', and 'Debate Etiquette / Brexit'), and for liberal and conservative there is no clear pattern ('EU Security / Defense', 'Debate Etiquette / Brexit', and 'Economic Development').

[TODO: add example?] [TODO: add interpretation?]

% (extreme)_right 'Trade Relations' (0.102); 'EU Security / Defense' (0.097); 'Economic Development' (0.071);
% christian_conservative 'EU Security / Defense' (0.091); 'Debate Etiquette / Brexit' (0.091); 'Economic Development' (0.091);
% liberal 'Debate Etiquette / Brexit' (0.129); 'Economic Development' (0.129); 'EU Security / Defense' (0.097);
% social_democratic 'Debate Etiquette / Brexit' (0.203); 'International Conflicts' (0.114); 'EU Security / Defense' (0.076);
% green 'EU Finances' (0.118); 'Disasters / Epidemics' (0.118); 'International Conflicts' (0.118);
% left 'International Conflicts' (0.132); 'EU Finances' (0.079); 'Debate Etiquette / Brexit' (0.079);

% Far-right parties ... 
% indicating that X and Y link migration with Z, 


\begin{figure}[ht]
\vskip 0.2in
\begin{center}
\centerline{\includegraphics[width=\columnwidth]{"fig/fig1_combined.pdf"}}
\caption{\textbf{Top:} Prevalence of selected topics in European Parliament debates over the past decade, as identified by LDA topic modeling. Proportions are computed by dividing by the total number of speeches per year. See repository for an interactive version with all topics.
\textbf{Bottom:} Absolute number of migration speeches by parliamentary group.}

\label{fig:fig1_lda}
\end{center}
\vskip -0.2in
\end{figure}
