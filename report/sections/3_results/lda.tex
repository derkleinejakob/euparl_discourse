% Salience of migration-related speech 
\subsection{Salience of migration-related speech}
\subsubsection{Who talks about migration?}
Figure [TODO] shows the prevalence of migration as parliamentary topic, compared to selected reference topics. 
It shows a consistent prevalence of the issue since 2014 with minor fluctuations, indicating a continuous interest in the topic. 
% TODO what parties talk about migration? do some parties talk more?

\subsubsection{Setting the agenda}
All parties exhibit signs of agenda-setting behaviour. However, it is significantly more pronounced for far-right parties, with a yearly average of $15.9\%$ of their migration-related speech occuring in debates where they are the only party talking about migration --- compared to $6.6\%$ for other parties on average (ANOVA $p<0.001$). Far-right parties bring up migration mostly in debates where the 'true' topic is related to foreign relations (top 'true' topics: 'Trade Relations', 'EU Security / Defense', or 'Economic Development') while progressive parties (social democratic, green, left) link it most to humanitarian situations ('Disasters / Epidemics', 'International Conflicts', and 'Debate Etiquette / Brexit'), while for liberal and conservative there is was clear pattern ('EU Security / Defense', 'Debate Etiquette / Brexit', and 'Economic Development'). There is no evidence for a systematic change over the years. 

% (extreme)_right 'Trade Relations' (0.102); 'EU Security / Defense' (0.097); 'Economic Development' (0.071);
% christian_conservative 'EU Security / Defense' (0.091); 'Debate Etiquette / Brexit' (0.091); 'Economic Development' (0.091);
% liberal 'Debate Etiquette / Brexit' (0.129); 'Economic Development' (0.129); 'EU Security / Defense' (0.097);
% social_democratic 'Debate Etiquette / Brexit' (0.203); 'International Conflicts' (0.114); 'EU Security / Defense' (0.076);
% green 'EU Finances' (0.118); 'Disasters / Epidemics' (0.118); 'International Conflicts' (0.118);
% left 'International Conflicts' (0.132); 'EU Finances' (0.079); 'Debate Etiquette / Brexit' (0.079);

% Far-right parties ... 
% indicating that X and Y link migration with Z, 


\begin{figure}[ht]
\vskip 0.2in
\begin{center}
\centerline{\includegraphics[width=\columnwidth]{"fig/fig1_combined.pdf"}}
\caption{\textbf{Top:} Prevalence of selected topics in European Parliament debates over the past decade, as identified by LDA topic modeling. Proportions are computed by dividing by the total number of speeches per year. See repository for an interactive version with all topics.
\textbf{Bottom:} Absolute number of migration speeches by parliamentary group.}

\label{fig:fig1_lda}
\end{center}
\vskip -0.2in
\end{figure}
