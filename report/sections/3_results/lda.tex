\subsection{Prevalence of migration speeches \& agenda setting}\label{LDAresults}

LDA organized all speeches into 30 interpretable and broadly coherent topics, with temporal trends in prevalence roughly consistent with salient world events, such as the outbreak of the Covid-19 pandemic in 2020 or the Russian invasion of Ukraine in 2022 (\Cref{fig:fig1_lda}). Although the proportion of migration-related speeches per year remained relatively stable, absolute counts peaked in 2015 and 2016, largely due to the use of written declarations (see \Cref{sec:methods}). 
% [TODO: insert ratios per party here?]

Migration-related speech rarely occurs in debates where only one party block addresses migration. Far right parties are a clear exception, with such instances acounting for $15.9\% \pm 7.7\%$ of their yearly mirgration-related speech, compared to $6.6\% \pm 0.6\%$ for other party blocks --- a significant difference (ANOVA $p<.001$). Far-right parties introduce migration mostly in debates where the `true' topic is ``Trade'', ``EU Security'', or ``Economic Development'', while left and green parties link it most to humanitarian issues (``International Conflicts'', ``EU Finances'', ``Disasters / Epidemics''). No clear pattern emerges for other parties, and there is no evidence of a systematic change over time.

% agenda setting in general low, but for extreme-right significantly more 
% extreme-right does it in certain context. for left & green in context X. 
% for them, very few examples, but could still indicate a difference in what topics the respective block links migration to
% While fluctuating, No systematic trend over the years 

% For most party blocks, there are only a handful of instances where they are the only group talking about migration. However for right-wing parties, there are $29 \pm 35.6$ (mean $ \pm$ std.) instances per year. While this number is low in absolute-terms, it still amounts to
%  --- compared to $6.6\%$ for other parties on average 

% hinting that there might be a strategic element to linking 

% % All parties exhibit signs of agenda-setting behaviour. However, it is significantly more pronounced for far-right parties, with a yearly average of 

% $15.9\%$ of their migration-related speech occuring in debates where they are the only parties talking about migration --- compared to $6.6\%$ for other parties on average (ANOVA $p<0.001$).  However the numbers are too little to assert this to be a trend. 

% their absoulte instances of agenda setting being low. 

% (extreme)_right 'Trade Relations' (0.102); 'EU Security / Defense' (0.097); 'Economic Development' (0.071);
% christian_conservative 'EU Security / Defense' (0.091); 'Debate Etiquette / Brexit' (0.091); 'Economic Development' (0.091);
% liberal 'Debate Etiquette / Brexit' (0.129); 'Economic Development' (0.129); 'EU Security / Defense' (0.097);
% social_democratic 'Debate Etiquette / Brexit' (0.203); 'International Conflicts' (0.114); 'EU Security / Defense' (0.076);
% green 'EU Finances' (0.118); 'Disasters / Epidemics' (0.118); 'International Conflicts' (0.118);
% left 'International Conflicts' (0.132); 'EU Finances' (0.079); 'Debate Etiquette / Brexit' (0.079);

% Far-right parties ... 
% indicating that X and Y link migration with Z, 


\begin{figure}[ht]
\vskip 0.2in
\begin{center}
\centerline{\includegraphics[width=\columnwidth]{"fig/fig1_combined.pdf"}}
\caption{LDA results. \textbf{Top:} Number of speeches for selected topics in EP debates, divided by total number of speeches per year. Topic labels were created manually based on most-frequent topic words. See repository for an interactive version with all topics.
\textbf{Bottom:} Absolute number of migration speeches by parliamentary group. Written speeches were discontinued in 2017.}

\label{fig:fig1_lda}
\end{center}
\vskip -0.2in
\end{figure}
