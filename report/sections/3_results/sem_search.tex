


\subsection{Similarities to Established Migration Narratives}\label{SemSearchResults}

Mean semantic similarities between speeches and two exemplary anti-immigration supernarratives, as well as the control prompt, are shown in \Cref{fig:fig4_search}. Mixed linear models reveal party differences for the three anti-immigration supernarratives ``immigration is a threat'', ``immigrants' culture is problematic', and ``immigration is a burden'', which are significantly higher for far-right speakers (all $s_{\cos} = 0.27$), compared to all other blocs (for threat and burden $s_{\cos} \in [0.23, 0.24]$; for problematic culture $s_{\cos} \in [0.24, 0.25]$, all $p < .003$). In contrast, no consistent party differences emerge for narratives framing \emph{immigrants as victims}, the \emph{Us vs. Them} supernarrative, or the control prompt. Across all supernarratives, no significant temporal trends are observed.

Out of all CHES expert party ratings, similarities to the three anti-immigration tropes are most correlated with anti-Islam rhetoric ($r \in [.39,.45]$) and salience of immigration in the party's agenda ($r = \in [.37,.40]$), followed by the salience of multiculturalism (for threat: $r = .37$; for burden: $r = .39$) and populist people vs.\ elite positioning (for problematic culture: $r = .36$). Us vs. Them narratives correlate most strongly with their conceptual counterparts, the people vs.\ elite rating ($r = .35$) and anti-elite salience ($r = .31$).