
\begin{figure*}[ht]
\vskip 0.2in
\begin{center}
\centerline{\includegraphics[width=\textwidth]{"fig/fig4_search.pdf"}}
\caption{Semantic similarity to selected migration narratives: temporal trajectories and top correlations to expert party ratings (CHES scores). Shadings indicate bootstrapped 95\% confidence intervals.}
\label{fig:fig4_search}
\end{center}
\vskip -0.2in
\end{figure*}

\subsection{Similarities to Established Migration Narratives}\label{SemSearchResults}

Mean semantic similarities between speeches and two exemplary anti-immigration supernarratives, as well as the constructed humanitarian prompt, are shown in \Cref{fig:fig4_search}. Mixed linear models revealed party differences for the three anti-immigration supernarratives ``immigration is a threat'', ``immigrants' culture is problematic', and ``immigration is a burden'', which were significantly higher for far-right speakers (all $d_{\cos} = 0.27$), compared to all other blocks (for threat \& problematic culture $d_{\cos} = 0.25$, for burden $d_{\cos} = 0.24$, all $p < .003$). In contrast, no consistent party differences emerged for narratives framing immigrants as victims, the populist Us vs. Them supernarrative, or the humanitarian control prompt. Across all supernarratives, no significant temporal trends were observed.

Out of all CHES expert party ratings, similarities to the three anti-immigration tropes were most correlated with anti-Islam rhetoric ($r = .39$–$.45$) and salience of immigration in the party's agenda ($r = .37$–$.40$), followed by the salience of multiculturalism (for threat: $r = .37$; for burden: $r = .39$) and populist people vs.\ elite positioning (for problematic culture: $r = .36$). Framing immigrants as victims showed weaker associations with related ratings. Us vs. Them narratives correlated most strongly with their conceptual counterparts, the people vs.\ elite rating ($r = .35$) and anti-elite salience ($r = .31$). No significant CHES score correlations were found for the control narrative.