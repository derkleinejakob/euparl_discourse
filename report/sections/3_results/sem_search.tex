
\begin{figure*}[ht]
\vskip 0.2in
\begin{center}
\centerline{\includegraphics[width=\textwidth]{"fig/fig4_search.pdf"}}
\caption{Temporal trajectories of semantic similarity to selected migration narratives for different party blocks. Party ratings taken from the Chapel Hill Expert survey. Shadings indicate bootstrapped 95\% confidence intervals.}
\label{fig:fig4_search}
\end{center}
\vskip -0.2in
\end{figure*}

\subsection{Similarities to Established Migration Narratives}\label{SemSearchResults}

Mean semantic similarities of speeches to two exemplary anti-immigration supernarratives and our constructed humanitarian prompt are visualized in \Cref{fig:fig4_search}. Mixed linear models revealed party differences for the three anti-immigration supernarratives ``immigration is a threat'', ``immigrants' culture is problematic', and ``immigration is a burden'', which were significantly higher for far-right speakers (all $d_{\cos} = 0.27$), compared to all other blocks (for threat \& problematic $d_{\cos} = 0.25$, for burden $d_{\cos} = 0.24$, all $p < .003$). For narratives falling under ``immigrants as victims'' or the populist ``Us vs. Them'', and the comparison ``Humanitarian'' prompt, no consistent differences emerged. For all supernarratives, no significant temporal trends were found.

Out of all expert party ratings, similarities to the three anti-immigration tropes were most correlated with anti-Islam rhetoric ($r = [.39,.45]$) and salience of immigration in their political agenda ($r = [.37,.40]$), on third position followed salience of multiculturalism ($r_{threat} = .37, r_{burden} = .39$) and populist people vs.\ elite ($r_{problem} = .36$). Immigrants as victims framing had lower correlations with similar dimensions, the us vs.\ them narrative category correlated most with its equivalent people vs.\ elite rating ($r = .35$). The humanitarian comparison narrative yielded no significant CHES score correlations.